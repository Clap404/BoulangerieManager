\chapter{Pré-requis et outils}

\section{Pré-requis matériels}
    \paragraph{}
        Cette solution logicielle peut théoriquement être hébergée sur tout
        support matériel de type ordinateur qui supporte les pré-requis
        logiciels.
        En d'autre terme notre application n'apporte aucune contrainte
        matérielle autre que celles imposées pour les outils que nous
        utilisons.
        Il est toutefois à noter qu'il est nécessaire que la machine abritant
        l'application dispose d'une connectivité réseau pour que l'application
        ait un sens.
    \paragraph{}
        En particulier, nous avons réalisé notre développement en utilisant des plate-forme x86\_64, ARMv6 et ARMv7. Ces architectures ne poseront de problème ni d'installation, ni d'utilisation, liés au matériel.

    \paragraph{}
        Pour son utilisation, la machine recevant les informations du serveur
        devra disposer de matériel permettant de se connecter à un réseau et
        de faire fonctionner un navigateur tel que décrit dans la section
        suivante.

\section{Pré-requis logiciels}
    \paragraph{}
        Cette solution logicielle nécessite pour son installation :
        \begin{description}
            \item[PHP version 5]\hfill \\
                Testé avec les versions 5.5.0 et 5.5.7 utilisant le moteur
                Zend Engine.
            \item[Le plugin php5-json]\hfill \\
                Nécessaire pour les distributions de PHP qui ne l'incluent pas.
            \item[Un serveur web proposant une interface CGI standard]\hfill \\
                Testé avec lightppd 1.4.33, apache 2.4.4 et nginx 1.4.4.
            \item[SQLite version 3]\hfill \\
                Testé avec la version 3.8.2.
            \item[Un système d'exploitation permettant d'utiliser les logiciels précités]\hfill \\
                Testé sous Microsoft Windows 8, Ubuntu 13.1, Arch Linux et
                FreeBSD 9.1.
        \end{description}

    \paragraph{}
        Pour son utilisation, cette application nécessite un navigateur web
        récent. Nous l'avons testée avec les navigateurs Chrome 31, Chromium
        34 et Firefox 26, 27 et 29.

\pagebreak
\section{Outils}
    \paragraph{}
        Ce logiciels fait usage des outils suivants.
        \begin{description}
            \item[Code Igniter 2.1.4]\hfill \\
                Framework PHP.
            \item[Foundation 5]\hfill \\
                Framework CSS.
            \item[jQuery 2.0.3]\hfill \\
                Dépendance de Foundation 5 que nous utilisons également.
            \item[Modernizr 2.6.2]\hfill \\
                Dépendance de Foundation 5.
            \item[FastClick 0.6.9]\hfill \\
                Dépendance de Foundation 5.
            \item[placeholder.js 3.0.2]\hfill \\
                Dépendance de Foundation 5.
            \item[bPopup 0.9.4]\hfill \\
                Permet l’utilisation de popups peu encombrants en javascript.
            \item[jqplot 1.0.8]\hfill \\
                Permet de tracer des graphiques en javascript.
            \item[le plugin jquery jDateTimePicker]\hfill \\
                Permet d'afficher des sélecteurs de date et heure avec masque
                de saisie.
        \end{description}