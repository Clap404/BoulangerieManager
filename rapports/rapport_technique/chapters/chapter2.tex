\chapter{Aperçu du projet}
\section{arborescence}
    \paragraph{}
        L'arborescence du projet se compose des répertoires user\_guide, system,
        uploads, assets et application.
        Le répertoire user\_guide contient le manuel d'utilisation du framework
        PHP Code Igniter.
        Le répertoire system contient les fichiers internes au framework,
        permettant de le faire fonctionner, par exemple des drivers de base
        de données, des librairies, etc.
        Le répertoire uploads est un dossier contenant les fichiers images
        temporaires téléversées par l'utilisateur.
        Sous un système de type UNIX, il est nécessaire que l'utilisateur dont
        PHP invoque l'identité pour évaluer le code source qui lui est
        transmis ait les droits d'écriture sur ce répertoire pour garantir le
        bon fonctionnement du système.
        Le répertoire assets est composé de fichiers statiques qui seront 
        servis par le serveur web, tels que les images, les fichiers javascript,
        et les fichiers css.
        Enfin le répertoire application est le répertoire contenant la
        configuration du projet, le code source PHP ainsi que la base de
        données SQLite.

    \paragraph{}
        Ce répertoire est lui même composé entre autres des répertoires 
        config, sqlite3, models, controllers et views.
        Les autres répertoires présents dans application sont d'un intérêt moins
        vital dans le cadre de ce rapport, vu qu'il s'agit de fichiers internes
        à Code Igniter.

    \paragraph{}
        Le répertoire config abrite les fichiers de configuration permettant de
        gérer les options du projet.
        L'identification à la base de données, par exemple y est faite.
        Le répertoire sqlite3 contient la base de données ainsi que le script de
        création de celle-ci.
        Il est à noter que ce répertoire et le fichier base de données 
        doivent, sous les systèmes de type UNIX posséder de droits identiques 
        à ceux du répertoire uploads suscité.
        Le répertoire models contient les classes de type modèle, permettant
        de faire le lien entre les données et le reste de l'application.
        Le répertoire controllers contient les classes de type contrôleur qui
        contiennent la logique applicative du logiciel.
        Enfin le répertoire views contient les classes de type vue qui
        affichent les données passées par le contrôleur dans le navigateur de
        l'utilisateur.


\pagebreak
\section{visualisation de l'arborescence}
    \dirtree{%
    .1 /.
    .1 user\_guide.
        .2 \ldots.
    .1 system.
        .2 \ldots.
    .1 uploads.
    .1 assets.
        .2 js.
            .3 stocks.
            .3 vendor.
            .3 foundation.
        .2 css.
        .2 images.
            .3 matprem.
            .3 produit.
    .1 application.
        .2 config.
        .2 models.
            .3 stocks.
            .3 commerce.
        .2 controllers.
            .3 stocks.
            .3 commerce.
        .2 views.
            .3 stocks.
            .3 templates.
            .3 fournisseurs.
            .3 commerce.
            .3 clients.
        .2 sqlite3.
        .2 \ldots.
    }
