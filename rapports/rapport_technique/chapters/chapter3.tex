\chapter{Détail des fichiers}

\paragraph{}
    Il est à noter qu'il existe un certain niveau d'incohérence dans les
    conventions de nommage de nos fichiers qui, même si l'on peut regretter
    sa présence, ne nuit pas au fonctionnement ni à la bonne compréhension
    des fonctionnalités du programme.

\section{Modèles, Vues et Contrôleurs}
    \subsection{Modèles}
    \paragraph{}
        Il existe dans notre projet les classes PHP de type modèle suivantes :
        \begin{description}
            \item[adresses\_m.php]\hfill \\
                Cette classe permet de faire le lien entre les données des
                adresses (ville, code postal, type de voie, ...) avec les
                différents contrôleurs impliqués : client et fournisseurs.
            \item[clients\_m.php]\hfill \\
                Cette classe permet de faire le lien entre les données des
                clients (nom, prénom, ...) avec le contrôleur dédié aux client
                .
            \item[fournisseurs\_m.php]\hfill \\
                Cette classe permet de faire le lien entre les données des
                fournisseurs (prix des fournitures, matières premières
                proposées) avec le contrôleur dédié aux fournisseurs.
            \item[stats\_m.php]\hfill \\
                Cette classe permet de récupérer de la base de données des
                données permettant de calculer des statistiques faisant appel
                à un trop grand nombre de tables pour appartenir aux autres
                modèles.
            \item[commande\_m.php]\hfill \\
                Cette classe permet de récupérer des informations le lien
                entre les données des commandes avec le contrôleur dédié aux
                commandes. contrôleurs.
            \item[vente\_m.php]\hfill \\
                Cette classe permet de faire le lien entre les données des
                ventes avec le contrôleurs dédié aux ventes.
            \item[matprem\_model.php]\hfill \\
                Cette classe permet de faire le lien entre les données des
                matières premières avec le contrôleur dédié aux matières
                premières.
            \item[produits\_model.php]\hfill \\
                Cette classe permet de faire le lien entre les données des
                produits avec le contrôleur dédié aux produits.
        \end{description}


    \subsection{Vues}
    \paragraph{}
        Nos vue sont les suivantes :
        \begin{description}
            \item[header.php]\hfill \\
                Cette vue contient les parties générales incluses dans chaque
                page, comme la barre de navigation.
            \item[footer.php]\hfill \\
                Cette vue contient le footer inclus dans chaque page.
            \item[welcome\_message.php]\hfill \\
                Cette vue contient la page d'accueil de notre application.
                Elle comprend un tableau de bord permettant à l'utilisateur
                de gagner en visibilité et d'accéder facilement à toutes les
                pages du site.
            \item[clients/clients\_v.php]\hfill \\
                Cette vue contient l'interface principale de l'interaction
                entre l'utilisateur et les données concernant les clients
            \item[clients/profil\_client\_v.php]\hfill \\
                Cette vue permet d'afficher le détail d'un client.
            \item[clients/add\_adresse\_v.php]\hfill \\
                Cette vue est rendue lorsque l'utilisateur veut ajouter une
                adresse d'un client dans la base de donnée.
            \item[clients/add\_joignable\_v.php]\hfill \\
                Cette vue est rendue lorsque l'utilisateur veut ajouter un
                numéro de téléphone auquel un client est joignable.
            \item[clients/add\_client\_v.php]\hfill \\
                Cette vue est rendue lorsque l'utilisateur veut ajouter un
                nouveau client dans la base de donnée.
            \item[clients/modif\_nom\_v.php]\hfill \\
                Cette vue est rendue lorsque l'utilisateur veut modifier le
                nom d'un client.
            \item[fournisseurs/fournisseurs\_v.php]\hfill \\  
                Cette vue contient l'interface principale de l'interaction
                entre l'utilisateur et les données concernant les fournisseurs
            \item[fournisseurs/profil\_fournisseur\_v.php]\hfill \\
                Cette vue permet d'afficher le détail d'un fournisseur.
            \item[fournisseurs/add\_adresse\_v.php]\hfill \\      
                Cette vue est rendue lorsque l'utilisateur veut ajouter une
                adresse d'un fournisseur dans la base de donnée.
            \item[fournisseurs/add\_joignable\_v.php]\hfill \\      
                Cette vue est rendue lorsque l'utilisateur veut ajouter un
                numéro de téléphone auquel un fournisseur est joignable.
            \item[fournisseurs/add\_fournisseur\_v.php]\hfill \\  
                Cette vue est rendue lorsque l'utilisateur veut ajouter un
                nouveau fournisseur dans la base de donnée.
            \item[fournisseurs/add\_modif\_matprem\_v.php]\hfill \\  
                Cette page est rendue lorsque l'utilisateur veut mettre à
                jour la liste des matières premières offertes par un
                fournisseur.
            \item[fournisseurs/modif\_nom\_v.php]\hfill \\
                Cette vue est rendue lorsque l'utilisateur veut modifier le
                nom d'un client.
            \item[informations/invendus\_v.php]\hfill \\
                Cette vue permet d'afficher les statistiques particulières
                concernant les produits invendus de l'utilisateur. La
                singularité de cette vue est motivée par l'importance de ces
                statistiques pour la profession à laquelle ce logiciel est
                destiné.
            \item[informations/stats\_v.php]\hfill \\
                Cette vue permet d'afficher des statistiques générales quant
                à l'activité de l'utilisateur
            \item[stocks/matprem\_v.php]\hfill \\
                Cette vue permet d'afficher une liste de matières premières.
            \item[stocks/matprem\_detail\_v.php]\hfill \\
                Cette vue permet d'afficher le détail d'une matière première.
            \item[stocks/produits\_v.php]\hfill \\
                Cette vue permet d'afficher la liste des produits enregistrés
                dans la base de données.
            \item[stocks/production.php]\hfill \\
                Cette vue permet d'accéder à l'interface de production de
                produits, centrale à la présente solution logicielle.
            \item[stocks/produits\_add.php]\hfill \\
                Cette vue permet d'afficher l'interface d'ajout de produit
                dans la base de données.
            \item[stocks/produits\_modif.php]\hfill \\
                Cette vue permet de mettre à jour les détails d'un produit
                dans la base de données.
        \end{description}

    \subsection{contrôleurs}
    \paragraph{}
        Nos classes PHP de type contrôleur sont les suivantes :
        \begin{description}
            \item[welcome.php]\hfill \\
                Ce contrôleur contient la logique permettant d'afficher le
                tableau de bord.
                Il communique avec différents modèle pour récupérer ses données
                et les affiche dans la vue welcome\_message.php.
            \item[fournisseurs.php]\hfill \\
                Ce contrôleur contient la logique permettant de manipuler les
                données liées aux fournisseurs, et de choisir la vue
                appropriée pour agir en tant qu'interface de communication 
                avec l'utilisateur.
            \item[adresses.php]\hfill \\
                Ce contrôleur contient la logique permettant de manipuler les
                adresses des clients et fournisseurs, et de choisir la vue
                appropriée pour les afficher.
            \item[informations/stats.php]\hfill \\
                Ce contrôleur contient la logique permettant de mettre au
                points les statistiques générales, et de les afficher en
                en rendant la vue informations/invendus\_v.php.
            \item[informations/invendus.php]\hfill \\
                Ce contrôleur contient la logique permettant de mettre au
                point les statistiques concernant les invendus et de les 
                afficher en randant la vue informations/stats\_v.php.
            \item[stocks/produits.php]\hfill \\
                Ce contrôleur contient la logique permettant de manipuler les
                données liées aux produits, et de choisir la vue appropriée
                pour agir en tant qu'interface de communication avec
                l'utilisateur.
            \item[stocks/matprem.php]\hfill \\
                Ce contrôleur contient la logique permettant de manipuler les
                données liées aux matières premières, et de choisir la vue
                appropriée pour agir en tant qu'interface de communication
                avec l'utilisateur.
            \item[stocks/production.php]\hfill \\
                Ce contrôleur contient la logique permettant de manipuler les
                données liées à la production, et de les afficher et modifier
                en utilisant la vue stocks/production.php comme interface.
            \item[clients.php]\hfill \\
                Ce contrôleur contient la logique permettant de manipuler les
                données liées aux clients, et de choisir la vue appropriée
                pour agir en tant qu'interface de communication avec
                l'utilisateur.
            \item[commerce/commande.php]\hfill \\
                Ce contrôleur contient la logique permettant de manipuler les
                données liées aux commandes, et de choisir la vue appropriée
                pour agir en tant qu'interface de communication avec
                l'utilisateur.
            \item[commerce/vente.php]\hfill \\
                Ce contrôleur contient la logique permettant de manipuler les
                données liées aux ventes, et de choisir la vue appropriée pour
                agir en tant qu'interface de communication avec l'utilisateur.
        \end{description}

\section{Base de données}
    \paragraph{}
        Utilisant le logiciel de gestion de base de données relationnelles
        SQLite, notre base de données est contenue dans un fichier facilement
        distribuable.
        Les fichiers en relation directe avec notre base de données sont les
        suivants :
        \begin{description}
            \item[BlMgr.sdb]\hfill \\
                La base de données en tant que telle, contenant toutes les
                données enregistrées par le logiciel.
            \item[generate.sql]\hfill \\
                Script de génération de la base de données, contenant son
                schéma en langage SQL.
            \item[populate.sql]\hfill \\
                Script de population de la base de données, contenant des
                données de test.
                Il n'est pas utile ni recommandé d'utiliser ce script en
                production.
        \end{description}
\section{Feuilles de style et script}
    \paragraph{}
        Le reste des fichiers déterminant le comportement de notre application
        est divisé en feuilles de styles et scripts javascript.
    \subsection{Feuilles de style}
        \paragraph{}
            Les feuilles de style permettant d'ajouter une touche graphique à
            cette application sont les suivantes :
        \begin{description}
            \item[foundation.css]\hfill \\
                Cette feuille de style est la version de développement du
                framework foundation.
                Son utilisation n'est conseillée que durant le développement : 
                en effet, pour la mise en production, nous recommandons
                l'utilisation de son équivalent "minifié", réduisant la taille
                du fichier sans perdre de contenu, plus apte à offrir de bonnes
                performances à l'application.
            \item[foundation.min.css]\hfill \\
                Version minifiée du fichier sus-décrit.
            \item[jquery.datetimepicker.css]\hfill \\
                Feuilles de style permettant de styliser le sélecteur de
                jour et heure utilisé dans la partie commande.
            \item[normalize.css]\hfill \\
                Feuille de style dont le framework foundation dépend.
            \item[style.css]\hfill \\
                Feuille de style propre à notre application, et développée
                pour celle ci.
        \end{description}
    \subsection{Scripts}
        \paragraph{}
            L'application utilise les scripts suivant pour permettre une plus
            grande facilité et convivialité d'utilisation pour les
            utilisateurs :
        \begin{description}
            \item[popup\_form.js]\hfill \\
                Ce script permet d'ouvrir un popup servant de formulaire pour
                différents composant de l'application
            \item[popup\_confirm.js]\hfill \\


            \item[vente.js]\hfill \\
                Ce script permet de rendre la saisie des ventes par 
                l'utilisateur plus agréable et interactive.
            \item[commande.js]\hfill \\
                Ce script permet de rendre la saisie des commandes par 
                l'utilisateur plus agréable et interactive.
            \item[stocks/matprem.js]\hfill \\
                Ce script permet de rendre l'interface de gestion des matières
                premières plus agréable et interactive.
            \item[stocks/matprem\_detail.js]\hfill \\
                Ce script permet de rendre la page de détail des matières
                premières plus agréable et interactive.
            \item[stocks/production.js]\hfill \\
                Ce script permet de rendre l'interface de production de
                produits plus agréable et interactive.
            \item[informations/invendus.js]\hfill \\
                Ce script permet de rendre la page des invendus plus facile
                à lire par la création de graphiques personnalisés.
            \item[foundation.min.js, modernizr.js, foundation/, vendor/]\hfill \\
                Ces fichiers et répertoires sont des dépendances du framework
                Foundation.
            \item[jquery.js]\hfill \\
                Ce fichier est la bibliothèque, utilisée par le framework
                Foundation et nous même pour rendre les pages web plus faciles
                à manipuler et permettre une plus grande compatibilité entre
                différents navigateurs.
            \item[jqplot/]\hfill \\
                Ce répertoire contient la bibliothèque jqplot, permettant
                de facilement créer des graphiques personnalisés.
            \item[bpopup.min.js]\hfill \\
                Ce fichier contient une bibliothèque permettant de créer des
                popups peu encombrants.
            \item[jquery.datetimepicker.js]\hfill \\
                Ce fichier contient une addition à la bibliothèque jQuery, lui
                permettant de créer des sélecteurs de date et heure.
        \end{description}


