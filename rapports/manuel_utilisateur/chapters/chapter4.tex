\chapter{Les informations}

\section{La page "Statistiques"}
Sur cette page sont présentés des statistiques sur les ventes.

\paragraph{}
Sur la page "Statistiques" se trouve un diagramme circulaire présentant la 
répartition des ventes de cette semaine en fonction du type de produits. Sous 
ce graphique on peut trouver une bouton "Depuis une semaine" et un autre bouton 
"Année Courante". Ces deux boutons permettent de naviguer entre le diagramme 
représentant les ventes de la semaine en cours ainsi que celui représentant la 
répartition des ventes de l'année courante.

\paragraph{}
Sur cette même page se trouve un bouton "Historique des ventes". Un clic sur ce 
bouton permet d'afficher un historique des ventes en fonction des années, 
grâce à une courbe. Sous cette courbe se situent alors des boutons représentant 
les différentes années que l'on peut observer avec des courbes.

\paragraph{}
En dessous de ce graphique, une barre déroulante "Tableau des ventes de la 
semaine" est présente. En cliquant sur cette barre, un tableau apparait 
représentant les ventes de la semaine. Ce tableau contient les noms des 
produits, le total des ventes depuis le début de la semaine ainsi que le 
pourcentage que cela représente par rapport aux autres ventes, sur la même 
période. Un autre clic sur cette barre masque le tableau à nouveau.



\section{La page "Invendus"}
Dans cette page sont présentés tous les invendus.

\paragraph{}
Sur la page "Invendus" se trouve un graphique présentant l'évolution des 
invendus durant les sept derniers jours. Il est possible de cliquer sur les 
jours pour lesquels il existe des invendus. En cliquant sur un jour, un 
diagramme circulaire s'affiche représentant la part de chaque produit dans les 
invendus du jour. Dans ce graphique se trouve une légende qui liste les 
différentes couleurs associées aux produits invendus. Un clic sur le bouton 
"Total" permet à tout moment de revenir au graphique représentant l'évolution 
sur les sept derniers jours.

\paragraph{}
En dessous de ce graphique, une barre déroulante "Tableau des invendus" est 
présente. En cliquant sur cette barre, un tableau apparait représentant les
invendus sur les sept derniers jours. Dans ce tableau sont renseignés les noms 
des produits, la quantité d'invendus ainsi que la date de production de ceux-ci.
Un second clic sur la barre déroulante masque le tableau à nouveau.
