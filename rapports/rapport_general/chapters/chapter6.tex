\chapter{Planning et répartition}

\section{Planning}
    \subsection{Planning théorique}
        \paragraph{}
            Voici le planning que nous avions mis au point au commencement du
            projet.
        \begin{center}
            \begin{tabular}{|l|l|}
                \hline
                18/10/2013 & MCD terminé. \\
                \hline
                28/10/2013 & Début du développement. \\
                \hline
                Novembre 2013 & Interface réalisée, base de donnée créée. \\
                \hline
                \multirow{2}{*}{Décembre 2013}
                    & Site fonctionnel, design proche du rendu final. \\
                    & Début de la phase de corrections de bugs. \\
                \hline
                07/01/2014 & Finalisation du développement. \\
                \hline
            \end{tabular}
        \end{center}
    \subsection{Planning réel}
        \paragraph{}
            Voici enfin le planning réel de notre développement.
            Nous avions au départ sous-estimé la charges de travail que nous
            aurions pour les autres cours, ce qui nous a conduit à mettre
            en place un planning très optimiste.
            Nous avons également eu besoin de refondre notre MCD au fur et à
            mesure que le projet se précisait, ce qui a également entraîné des
            retards.
            Enfin nous avons continué à développer la solution logicielle
            jusqu'en janvier, ce qui nous a empêché d'avoir une plage
            uniquement aux tests comme nous l'avions prévu originellement.
        \begin{center}
            \begin{tabular}{|l|l|}
                \hline
                \multirow{2}{*}
                      {18/10/2013}
                    & Première version du MCD terminée. \\
                    & Choix des outils effectué. \\
                \hline
                    20/10/2013 & Début du développement. \\
                \hline
                    Novembre 2013 & Deuxième version du MCD terminée. \\
                \hline
                \multirow{2}{*}
                      {Décembre 2013}
                    & Troisième version du MCD terminée. \\
                    & Début des tests en parallèle avec le développement \\
                \hline
                    07/01/2014 & Finalisation du développement. \\
                \hline
            \end{tabular}
        \end{center}

\section{Répartition des tâches}
    \paragraph{}
        Voici la façon dont nous nous avons pensé la répartition des tâches
        au début du projet.
        Ne sont pas listées les parties que nous avons faites
        ensemble, telles que la modélisation et la création de la base
        de données.

    \begin{center}
        \begin{tabular}{|l|l|}
            \hline
                Jérémy Autran & Parties ventes et commandes. \\
            \hline
                Florian Barrois & Parties statistiques et invendus. \\
            \hline
                François-Xavier Béligat & Partie produits. \\
            \hline
                Maxime Chenot & Partie graphique. \\
            \hline
                Arnaud Colin & Parties statistiques et invendus. \\
            \hline
                Benoît Houdayer & Parties fournisseurs et clients. \\
            \hline
                Anthony Ruhier & Partie matières premières. \\
            \hline
        \end{tabular}
    \end{center}

        \paragraph{}
        Les tâches ont été réparties différemment suite aux problèmes d'
        organisation que nous avons rencontrées lors du développement du projet.
        Voici la répartition réelle en fin de projet :


    \begin{center}
        \begin{tabular}{|l|l|}
            \hline
                Jérémy Autran & Parties ventes et commandes. Page d'accueil.\\
            \hline
                Florian Barrois & Parties statistiques. \\
            \hline
                François-Xavier Béligat & Partie produits. \\
            \hline
                Maxime Chenot & Organisation graphique et design. \\
            \hline
                Arnaud Colin & Parties statistiques. \\
            \hline
                Benoît Houdayer & Parties fournisseurs et clients. \\
            \hline
                Anthony Ruhier & Partie matières premières. Partie invendus.
                                 Partie statistiques.\\
            \hline
        \end{tabular}
    \end{center}



