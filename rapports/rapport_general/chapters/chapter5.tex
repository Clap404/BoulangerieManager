\chapter{Problèmes rencontrés}


\section{Problèmes techniques}
    \paragraph{}
        Nous avons eu quelques difficultés aux premiers abords à utiliser les 
        outils logiciels que nous avons utilisé, notamment CodeIgniter, notre
        framework PHP.
        En guise d'illustration, nous avons essayé d'utiliser ce framework 
        sous le paradigme du Model-View-Controller, mais après analyse, il
        utilise non pas ce dernier mais se trouve être plus proche du
        paradigme Model-View-Presenter, légèrement différent dans le rôle
        des éléments mis en jeu, notamment, dans ce paradigme, la vue n'est
        jamais en contact direct avec le modèle.
        Heureusement, nous avons réussi à nous y adapter après avoir étudié
        la documentation et lu des explications sur internet.
        
\section{Problèmes de gestion}
    \paragraph{}
        La gestion du projet ne s'est pas déroulée de façon aussi fluide que 
        l'on aurait pu espérer. En effet, nous sommes parti sur un modèle de 
        collaboration basé sur l'implication de chacun et nous sommes rendus
        compte que cela ne marchait pas. Il est dans la pratique plus difficile
        que prévu de laisser l'équipe se tenir au courant des avancées du
        projet de façon autonome, et illusoire d'espérer une avancée 
        satisfaisante de la part des collaborateurs qui n'ont pas suivi le 
        déroulement du projet et s'y sentent perdus. Nous y voyons comme 
        solution une meilleure communication au sein du groupe par des biais
        plus directs. En effet, sur la fin du projet nous nous sommes mis à
        utiliser les fonctionnalités propre à github, notre solution 
        d'hébergement de code, telles que le mécanisme des issues ou celui des
        commentaires de commits. Cette solution s'est révélée plus efficace que
        notre solution précédente d'utiliser un canal IRC privé, parce qu'elle
        offre une base centralisée d'information, consultable à tout moment,
        et permet d'envoyer des notifications sous forme de courriel aux
        personnes concernées par des commentaires particuliers.

    \paragraph{}
        Nous avons également senti des problèmes d'organisation que l'on 
        pourrait qualifier de goulot d'étranglement.
        En effet, il nous est arrivé de devoir attendre certaines
        fonctionnalités de nos collaborateurs pour pouvoir débuter les nôtres.
        Ce problème a eu pour effet de nous retarder dans le développement et
        nous n'avons pas trouvé de solution adaptée pour y remédier en milieu
        de projet.
        Pour des projets à venir, nous pourrons tenter d'éviter ce problème en
        définissant dès le début du projet une interface de programmation
        commune, dans la lignée de la programmation par contrat.
        Ceci pourrait permettre d'utiliser des fonctionnalités assignées à 
        d'autres membres du groupe sans qu'elles n'aient été finalisées, et de
        contribuer à sa complétion en ne se souciant que des priorités propres
        à ce qui nous a été assigné.

