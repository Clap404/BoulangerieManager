\chapter{Analyse}

\section{Données}

\section{Outils graphiques}
    \paragraph{}
        Nous avons pris le choix de ne pas nous tourner vers une imagerie de
        proche de l'univers de la boulangerie, mais d'utiliser des éléments
        graphiques plus modernes pour plusieurs raisons.
    \paragraph{}
        Tout d'abord, cela met l'accent sur l'universalité du projet :
        il s'agit bien d'un logiciel de gestion de boulangerie, mais plus
        généralement, il pourrait servir de base à la gestion de n'importe
        quel commerce artisan, dont l'objet serait d'acheter des matières
        premières, de les transformer puis de vendre le produit de cette
        transformation.
    \paragraph{}
        Ensuite, il nous a paru intéressant de diriger nos efforts vers un
        produit facile à utiliser par n'importe quel utilisateur.
        Ainsi, nous imaginons que le parallèle existant entre notre charte
        graphique et l'interface métro de Microsoft Windows 8 permettra à un
        utilisateur de prendre en main notre produit plus facilement.
        Nous avons, dans cette démarche, également utilisé un code couleur
        adapté à faire passer le message que nous voulons véhiculer.
        Par exemple, un produit sélectionné dans la partie vente ou commande
        est marqué par un fond vert, un produit ou une matière première non
        disponible sera marquée par un fond rouge, un objet non sélectionné
        n'aura pas de couleur de fond particulier autre que le couleur de fond
        du reste de la page.
        Nous avons également opté de permettre à l'utilisateur de téléverser
        des images de produit et matières premières lui permettant de naviguer
        des listes plus naturellement sans devoir lire les titres.