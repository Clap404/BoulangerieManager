\chapter{Analyse}

\section{Données}
    \paragraph{}
        Afin de trouver un modèle adapté à notre problématique nous avons étudié
        les besoins spécifiques de la profession de boulanger.
        Nous en avons déduit les informations que nous devions être capables de
        traiter.
    \subsection{commerce}
        Le commerce se divise selon notre analyse en deux parties bien
        distinctes. 
        Les ventes, effectuées par des clients en magasin et les commandes,
        effectuées éventuellement par téléphone, par des clients désireux d'être
        livrés dans un avenir proche pour des produits qui seront réservés.
        Les ventes doivent prendre en compte les disponibilités du jour,
        permettant de savoir si un produit a été produit et n'a pas été réservé
        pour une commande livrable le jour même.
        Les commandes en revanche peuvent être effectuées sans se soucier des 
        disponibilités : le boulanger s'occupera de produire les produits
        réservés le jour de la livraison.
    \subsection{stock}
        Ici encore nous avons mis en évidence deux types de stocks : les stocks
        de matières premières et les stocks de produits. Ces deux catégories 
        sont intimement liées puisqu'à la création d'un produit, on utilise
        les stocks des matières premières. Elles ont cependant des différences
        notables.
        Par exemple les stocks de matières premières ne peuvent se remplir
        qu'en passant des commandes à des fournisseurs.
        Les stocks de produits en revanche résultent de la transformation de
        matières premières.
        Les produits ont aussi la caractéristique d'être éphémères.
        En effet, nous avons considéré qu'un produit ne pouvait se conserver
        plus longtemps qu'une journée, suivant le cas général des artisans 
        boulangers.
        Il est donc nécessaire pour un boulanger d'éviter les invendus en
        produisant suffisamment pour répondre à la demande de ses clients mais
        suffisamment peu pour éviter de gaspiller ses précieuses matières 
        premières en les transformant en produits invendus.
        Nous avons imaginé un cycle d'approvisionnement des matières premières
        en flux tendu et n'avons pas jugé utile de considérer leur péremption.
        Il aurait toutefois été possible de le faire en considérant les
        ressources les plus périssables, telles que les œufs.
    \subsection{acteurs}
        Nous avons identifié deux types d'acteurs jouant un rôle important dans
        la boulangerie : les fournisseurs et les clients.
        Les fournisseurs sont des personnes, souvent morales, qu'il s'agisse
        de petites et moyennes entreprises, de coopératives ou autre, qui
        fournissent des matières premières à la boulangerie.
        Les clients, en revanche, sont les personnes, souvent physiques, qui
        achètent les produits de la boulangerie.
        Nous avons décidé, afin de refléter la réalité de la profession, de 
        ne garder de trace que des clients voulant passer des commandes, et des
        clients qui désireraient bénéficier d'un programme de fidélisation
        éventuellement mis en place par la boulangerie.
        Il est donc possible pour un client d'effectuer des ventes en magasin
        sans dévoiler son identité.
        nous avons en revanche prévu qu'on client passant une commande devrait
        s'identifier auprès du personnel de la boulangerie et donner, son
        adresse de livraison et un numéro de téléphone par lequel il pourra être
        joignable.



\section{Outils graphiques}
    \paragraph{}
        Nous avons pris le choix de ne pas nous tourner vers une imagerie de
        proche de l'univers de la boulangerie, mais d'utiliser des éléments
        graphiques plus modernes pour plusieurs raisons.
    \paragraph{}
        Tout d'abord, cela met l'accent sur l'universalité du projet :
        il s'agit bien d'un logiciel de gestion de boulangerie, mais plus
        généralement, il pourrait servir de base à la gestion de n'importe
        quel commerce artisan, dont l'objet serait d'acheter des matières
        premières, de les transformer puis de vendre le produit de cette
        transformation.
    \paragraph{}
        Ensuite, il nous a paru intéressant de diriger nos efforts vers un
        produit facile à utiliser par n'importe quel utilisateur.
        Ainsi, nous imaginons que le parallèle existant entre notre charte
        graphique et l'interface métro de Microsoft Windows 8 permettra à un
        utilisateur de prendre en main notre produit plus facilement.
        Nous avons, dans cette démarche, également utilisé un code couleur
        adapté à faire passer le message que nous voulons véhiculer.
        Par exemple, un produit sélectionné dans la partie vente ou commande
        est marqué par un fond vert, un produit ou une matière première non
        disponible sera marquée par un fond rouge, un objet non sélectionné
        n'aura pas de couleur de fond particulier autre que le couleur de fond
        du reste de la page.
        Nous avons également opté de permettre à l'utilisateur de téléverser
        des images de produit et matières premières lui permettant de naviguer
        des listes plus naturellement sans devoir lire les titres.