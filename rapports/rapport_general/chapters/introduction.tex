\chapter*{Introduction}
\addcontentsline{toc}{chapter}{Introduction}

\paragraph{}
    Dans le cadre de notre projet de gestion de bases de données,
    nous avons pris l'initiative de nous assigner un sujet très concret,
    proche de nous aussi bien en tant que consommateurs que du monde des PME et
    de entrepreneuriat, que nous avons beaucoup de chances de servir en tant
    qu'analystes programmeurs après obtention de notre diplôme.
    Ainsi, nous nous sommes laissé tenter par une problématique à la fois très
    française, mais applicable au domaine du commerce en général.
    Nous avons imaginé une solution de gestion de boulangerie.
    Ce sujet réunit à la fois un point de vue très concret sur les solutions
    logicielles que nous serons amenés à réaliser dans notre vie
    professionnelle, aussi bien que très simple au niveau des règles de
    gestion : nous disposons de fournisseurs qui nous vendent des matières
    premières, celles ci sont transformées en produits très périssable et ces
    derniers sont vendus le jour même en espérant avoir le moins de pertes
    possible.
\paragraph{}
    pour mettre la lumière sur le projet et notre réalisation, nous allons 
    d'abord présenter ce sujet dans les détails et nous plonger dans le cahier
    des charges de notre projet. Ensuite, nous nous lancerons dans une analyse
    des données mises en jeu et des moyens graphiques utilisés pour
    les rendre exploitables.
    Nous joindront à cette analyse une illustration du travail réalisé en
    quelques images et nous attaquerons enfin à une analyse de la façon dont le
    projet a été géré. Ainsi, nous parlerons des problèmes que nous avons pu
    rencontrer, de la planification que nous nous sommes fixées aussi bien au
    niveau du temps que des personnes : nous vous offrirons un aperçu de la 
    planification théorique fixée en début de projet ainsi que la planification
    réelle, telle qu'elle a eu lieu. Nous terminerons bien entendu par une
    brève conclusion qui mettra en avant les points importants de notre projet.
