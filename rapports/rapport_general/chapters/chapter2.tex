\chapter{Cahier des charges}

\section{Description du produit}
    \paragraph{}
        Le produit sera un site web, écrit dans les langages suivants.
        \begin{itemize}
            \item PHP
            \item SQL
            \item HTML
            \item Javascript
            \item CSS
        \end{itemize}
        Se produit sera livré sous forme d'archive contenant le code source,
        ainsi qu’une base de données prête à l’emploi permettant d'effectuer
        les opérations décrites dans la section suivante.

\section{Description des fonctionnalités}
    \subsection{Outil de gestion des fournisseurs}
        \paragraph{}
            Nous devrons garder les informations indispensables à la prise de
            contact avec les fournisseurs, tels que leurs noms, leurs adresses,
            et leurs numéros de téléphone, mais aussi les informations
            permettant de faire des décisions de gestion, comme le prix offert
            par chaque fournisseur pour les matières premières qu'ils proposent.
    \subsection{Outil de gestion des achats de matières premières}
        \paragraph{}
            Nous calculerons de façon efficace la disponibilité des matières
            premières, avertirons l'utilisateur lorsque son stock de matières
            premières est proche de devenir vide et lui offrirons  des
            informations sur le prix de revient permettant de faire un choix
            éclairé quant aux fournisseurs potentiels.
    \subsection{Outil de gestion de la production et des produits}
        \paragraph{}
            Nous permettrons à l'utilisateur de déclarer sa production du jour,
            et utiliserons ces valeurs pour calculer les disponibilités, afin de
            lui permettre d'avoir à tout moment une idée précise de la santé de
            ses stocks et d'adapter sa production en fonction.
    \subsection{Outil de gestion de la vente de ces produits}
        \paragraph{}
            Nous permettrons à l'utilisateur d'enregistrer des commandes et des
            ventes en magasin avec une interface simple, et d'en obtenir le
            coût détaillé.
            Nous lui offrirons également la possibilité de garder une trace
            visuelle des prochaine commandes à livrer, ainsi que les produits
            les constituant.
    \subsection{Outil de gestion des clients}
        \paragraph{}
            Nous garderons une trace des clients ayant commandé ou acheté des
            produits, ainsi que les données permettant de les contacter,
            afin de permettre à l'utilisateur de les contacter facilement en
            cas de besoin d'informations supplémentaires au sujet de leur
            commande ou en cas de souhait de la part de l'utilisateur de mettre
            en place un mécanisme de fidélisation par la récompense des
            meilleurs clients.
    \subsection{Outil de visualisation de statistiques pertinentes}
        \paragraph{}
            Nous afficherons des statistiques sur les ventes et les invendus qui
            permettront à l'utilisateur de moduler efficacement sa production et
            ses commandes de matières premières en fonction de la demande.